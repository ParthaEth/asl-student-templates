\chapter*{Abstract}
\addcontentsline{toc}{chapter}{Abstract}
%\chapter*{Zusammenfassung}
%\addcontentsline{toc}{chapter}{Zusammenfassung}

Optimization methods based on selected frames (key-frames/key-vertex) are widely used because a globally optimal solution to the Simultaneous Localization and Mapping (SLAM) problem is computationally intensive especially for large maps. Main drawback of key-frame based techniques is that they discard information associated with non-key-frames. Constrained Key-frame based Localization and Mapping (CKLAM) is a technique that attempts to retain such information as much as possible. In this thesis, a method that extends CKLAM to form Relative-CKLAM (RCKLAM) is presented, which makes it particularly suitable in the multiple agent aided mapping context. CKLAM retains information from non-key-frames by representing it as a Gaussian distribution around each key-frame (key-vertex). While it has been shown that this method is efficient, in certain circumstances such as change of global frame of reference and map merging, it suffers from added computational penalty associated with re-projection of the previously retrieved information. In RCKLAM we address these issues by representing the same information computed in CKLAM in a local vertex's reference frame rather than the global frame. It has been shown that RCKLAM performs as good as CKLAM in terms of solution quality while overcoming all the above discussed limitations on both simulated and real visual inertial maps (vi-map). Relative CKLAM presented in this work improves scalability of full batch optimization also known as the Bundle Adjustment (BA) step of the SLAM problem by projecting the information available in the non-key-frames onto the relative transformation between two consecutive key-frames. Unlike CKLAM which projects this information onto the key frames itself RCKLAM achieves flexibility such as, global frame independence and better linearization-trust region. Thereby making it more suitable tool for a multiple agent aided mapping framework.

We demonstrate the advantages and disadvantages of CKLAM method with both simulated and real world data and also show the advantages gained by deploying RCKLAM. Performance and time complexity of all the algorithms have been compared and visualized. 